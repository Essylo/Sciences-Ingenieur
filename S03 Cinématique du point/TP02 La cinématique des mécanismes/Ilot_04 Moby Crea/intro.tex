\newcommand{\id}{57}
\newcommand{\nom}{La cinématique des mécanismes}
\newcommand{\sequence}{03}
\newcommand{\num}{02}
\newcommand{\type}{TP}
\newcommand{\descrip}{Lois E/S de fermeture géométrique et cinématique. Simulation du comportement de modèles. Proposer des lois de commande en fonction d'exigences. Présenter les modèles acausaux}
\newcommand{\competences}{Mod2-C10-1: Modèle de solide indéformable \\ &  Mod2-C11: Modélisation géométrique et cinématique des mouvements entre solides indéformables \\ &  Rés-C1: Loi entrée sortie géométrique et cinématique \\ &  Rés-C6: Utilisation d'un solveur ou d'un logiciel multi physique \\ &  Com1-C1: Différents descripteurs introduits dans le programme \\ &  Com2-C4: Outils de communication}
\newcommand{\nbcomp}{6}
\newcommand{\systemes}{Moby Crea}
\newcommand{\systemessansaccent}{Moby Crea}
\newcommand{\ilot}{4}
\newcommand{\ilotstr}{04}
\newcommand{\dossierilot}{\detokenize{Ilot_04 Moby Crea}}
\newcommand{\imageun}{Moby_Crea}

\newcommand{\urlsysteme}{\href{https://www.costadoat.fr/systeme/55}{Ressources système}}
\newcommand{\matlabsimscape}{\href{https://github.com/Costadoat/Sciences-Ingenieur/raw/master/Systemes/Moby Crea/mobycrea_complet.zip}{Modèle Simulink complet}}
\newcommand{\matlabsimscapei}{\href{https://github.com/Costadoat/Sciences-Ingenieur/raw/master/Systemes/Moby Crea/Mobycrea_Simscape.zip}{Modèle Simscape}}
\newcommand{\experimental}{\href{https://github.com/Costadoat/Sciences-Ingenieur/raw/master/Systemes/Moby Crea/MobyCrea_experimental.zip}{Analyse de résultats expérimentaux}}
\newcommand{\miseenoeuvre}{\href{https://github.com/Costadoat/Sciences-Ingenieur/raw/master/Systemes/Moby Crea/MobyCrea_MO/MobyCrea_MO.pdf}{Mise en oeuvre}}
\newcommand{\schemacinematique}{MobyCrea_cinematique}
\newcommand{\schemabloc}{MobyCrea_schema_bloc}
