\newcommand{\id}{98}
\newcommand{\nom}{La cinématique des mécanismes}
\newcommand{\sequence}{06}
\newcommand{\num}{01}
\newcommand{\type}{TP}
\newcommand{\descrip}{Lois E/S de fermeture géométrique et cinématique. Simulation du comportement de modèles. Proposer des lois de commande en fonction d'exigences. Présenter les modèles acausaux}
\newcommand{\competences}{Mod2-C10-1: Modèle de solide indéformable \\ &  Mod2-C11: Modélisation géométrique et cinématique des mouvements entre solides indéformables \\ &  Rés-C1: Loi entrée sortie géométrique et cinématique \\ &  Rés-C6: Utilisation d'un solveur ou d'un logiciel multi physique \\ &  Com1-C1: Différents descripteurs introduits dans le programme \\ &  Com2-C4: Outils de communication}
\newcommand{\nbcomp}{6}
\newcommand{\systemes}{Barrière}
\newcommand{\ilot}{2}
\newcommand{\imageun}{Barriere}
