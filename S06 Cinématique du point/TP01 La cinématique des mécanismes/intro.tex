\newcommand{\id}{57}
\newcommand{\nom}{La cinématique des mécanismes}
\newcommand{\sequence}{06}
\newcommand{\num}{01}
\newcommand{\type}{TP}
\newcommand{\descrip}{Lois E/S de fermeture géométrique et cinématique. Simulation du comportement de modèles. Proposer des lois de commande en fonction d'exigences. Présenter les modèles acausaux}
\newcommand{\competences}{Mod2-C10-1: Modèle de solide indéformable \\ &  Mod2-C11: Modélisation géométrique et cinématique des mouvements entre solides indéformables \\ &  Rés-C1: Loi entrée sortie géométrique et cinématique \\ &  Rés-C6: Utilisation d'un solveur ou d'un logiciel multi physique \\ &  Com1-C1: Différents descripteurs introduits dans le programme \\ &  Com2-C4: Outils de communication}
\newcommand{\nbcomp}{6}
\newcommand{\systemes}{Maxpid}
\newcommand{\ilot}{1}
\newcommand{\ilotstr}{01}
\newcommand{\dossierilot}{Ilot_01 Maxpid}
\newcommand{\imageun}{Maxpid}

\newcommand{\urlsysteme}{\href{https://www.costadoat.fr/systeme/26}{Ressources système}}
\newcommand{\scilabxcos}{\href{https://github.com/Costadoat/Sciences-Ingenieur/raw/master/Systemes/Maxpid/Maxpid_complet.zcos}{Modèle complet du  Maxpid}}
\newcommand{\matlabsimscape}{\href{https://github.com/Costadoat/Sciences-Ingenieur/raw/master/Systemes/Maxpid/Maxpid_Simscape.zip}{Modèle Simscape}}
\newcommand{\solidworks}{\href{https://github.com/Costadoat/Sciences-Ingenieur/raw/master/Systemes/Maxpid/Maxpid_Solidworks.zip}{Modèle Solidworks}}
\newcommand{\miseenoeuvre}{\href{https://github.com/Costadoat/Sciences-Ingenieur/raw/master/Systemes/Maxpid/Maxpid_MO/Maxpid_MO.pdf}{Mise en oeuvre}}
\newcommand{\edrawings}{\href{https://github.com/Costadoat/Sciences-Ingenieur/raw/master/Systemes/Maxpid/Maxpid.EASM}{Fichier eDrawing}}
\newcommand{\experimental}{\href{https://github.com/Costadoat/Sciences-Ingenieur/raw/master/Systemes/Maxpid/Maxpid_experimental.zip}{Analyse de résultats expérimentaux}}
\newcommand{\schemacinematique}{Maxpid_cinematique}
