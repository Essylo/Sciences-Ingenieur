\newcommand{\id}{85}
\newcommand{\nom}{Structure des SLCI}
\newcommand{\sequence}{02}
\newcommand{\num}{02}
\newcommand{\type}{TP}
\newcommand{\descrip}{Modélisation de la structure d'un SLCI. Boucles ouvertes et boucles fermées.}
\newcommand{\competences}{Mod2-C4: Systèmes linéaires continus invariants asservis \\ &  Mod2-C20: Modélisation des systèmes asservis}
\newcommand{\nbcomp}{2}
\newcommand{\systemes}{Cordeuse}
\newcommand{\systemessansaccent}{Cordeuse}
\newcommand{\ilot}{1}
\newcommand{\ilotstr}{01}
\newcommand{\dossierilot}{\detokenize{Ilot_01 Cordeuse}}
\newcommand{\imageun}{Cordeuse}

\newcommand{\urlsysteme}{\href{https://www.costadoat.fr/systeme/48}{Ressources système}}
\newcommand{\videoavi}{\href{https://github.com/Costadoat/Sciences-Ingenieur/raw/master/Systemes/Cordeuse/Corder_raquette_de_tennis.avi}{Comment corder une raquette de tennis}}
\newcommand{\videoavii}{\href{https://github.com/Costadoat/Sciences-Ingenieur/raw/master/Systemes/Cordeuse/Demonstration_cordeuse.avi}{Démonstration de l'utilisation d'une cordeuse}}
\newcommand{\schemabloc}{Cordeuse_schema_bloc}
