\newcommand{\id}{94}
\newcommand{\nom}{Liaisons mécaniques}
\newcommand{\sequence}{04}
\newcommand{\num}{02}
\newcommand{\type}{TP}
\newcommand{\descrip}{Modélisation d'un solide. Comportement des liaisons mécaniques. Modéliser les mécanismes du laboratoire par un schéma cinématique, paramétré.}
\newcommand{\competences}{A3-C4: Analyse d'architecture et de comportement \\ &  Mod1-C1: Isolement d'un solide ou d'un système de solides \\ &  Mod2-C10-1: Modèle de solide indéformable \\ &  Mod2-C11: Modélisation géométrique et cinématique des mouvements entre solides indéformables \\ &  Mod2-C12: Modélisation cinématique des liaisons entre solides \\ &  Mod2-C15: Modélisation des actions mécaniques \\ &  Rés-C6: Utilisation d'un solveur ou d'un logiciel multi physique \\ &  Com1-C1: Différents descripteurs introduits dans le programme \\ &  Com2-C4: Outils de communication}
\newcommand{\nbcomp}{9}
\newcommand{\systemes}{Plateforme Stewart}
\newcommand{\ilot}{2}
\newcommand{\imageun}{Plateforme}

\newcommand{\urlsysteme}{\href{https://www.costadoat.fr/systeme/57}{Ressources système}}
\newcommand{\matlabsimscape}{\href{https://github.com/Costadoat/Sciences-Ingenieur/raw/master/Systemes/Plateforme Stewart/Plateforme_Stewart_Simscape.zip.zip}{Modèle Simscape}}
\newcommand{\solidworks}{\href{https://github.com/Costadoat/Sciences-Ingenieur/raw/master/Systemes/Plateforme Stewart/Plateforme_Stewart_Solidworks.zip.zip}{Modèle Solidworks}}
