\newcommand{\id}{55}
\newcommand{\nom}{Théorie des mécanismes}
\newcommand{\sequence}{04}
\newcommand{\num}{02}
\newcommand{\type}{TP}
\newcommand{\descrip}{Le système de solides, la théorie des mécanismes. Hyperstatisme et mobilités. Proposer des solutions pour le rendre isostatique et justifier les choix de conception}
\newcommand{\competences}{A3-C6: Transmetteurs de puissance \\ &  Mod2-C12: Modélisation cinématique des liaisons entre solides \\ &  Mod2-C14: Modèle cinématique d'un mécanisme \\ &  Conc1-C2: Démarche de conception appliquée aux fonctions techniques \\ &  Conc1-C3: Les fonctions techniques \\ &  Com2-C4: Outils de communication}
\newcommand{\nbcomp}{6}
\newcommand{\systemes}{Barrière Sympact}
\newcommand{\ilot}{4}
\newcommand{\ilotstr}{04}
\newcommand{\dossierilot}{\detokenize{Ilot_04 Barrière Sympact}}
\newcommand{\imageun}{Barriere}

\newcommand{\urlsysteme}{\href{https://www.costadoat.fr/systeme/49}{Ressources système}}
\newcommand{\solidworks}{\href{https://github.com/Costadoat/Sciences-Ingenieur/raw/master/Systemes/Barriere Sympact/Barriere_Solidworks.zip}{Modèle Solidworks}}
\newcommand{\matlabsimscape}{\href{https://github.com/Costadoat/Sciences-Ingenieur/raw/master/Systemes/Barriere Sympact/Barriere_Simscape.zip}{Modèle Simscape}}
\newcommand{\edrawings}{\href{https://github.com/Costadoat/Sciences-Ingenieur/raw/master/Systemes/Barriere Sympact/Barriere.EASM}{Modèle eDrawings}}
\newcommand{\videoavi}{\href{https://github.com/Costadoat/Sciences-Ingenieur/raw/master/Systemes/Barriere Sympact/Utilisation_barriere_1.avi}{Utilisation de la barrière}}
\newcommand{\videoavii}{\href{https://github.com/Costadoat/Sciences-Ingenieur/raw/master/Systemes/Barriere Sympact/Utilisation_barriere_2.avi}{Utilisation de la barrière avec télécommande}}
\newcommand{\experimental}{\href{https://github.com/Costadoat/Sciences-Ingenieur/raw/master/Systemes/Barriere Sympact/Barriere_experimental.zip}{Analyse de résultats expérimentaux}}
\newcommand{\schemacinematique}{Barriere_cinematique}
