\newcommand{\id}{55}
\newcommand{\nom}{Théorie des mécanismes}
\newcommand{\sequence}{04}
\newcommand{\num}{02}
\newcommand{\type}{TP}
\newcommand{\descrip}{Le système de solides, la théorie des mécanismes. Hyperstatisme et mobilités. Proposer des solutions pour le rendre isostatique et justifier les choix de conception}
\newcommand{\competences}{A3-C6: Transmetteurs de puissance \\ &  Mod2-C12: Modélisation cinématique des liaisons entre solides \\ &  Mod2-C14: Modèle cinématique d'un mécanisme \\ &  Conc1-C2: Démarche de conception appliquée aux fonctions techniques \\ &  Conc1-C3: Les fonctions techniques \\ &  Com2-C4: Outils de communication}
\newcommand{\nbcomp}{6}
\newcommand{\systemes}{Cordeuse}
\newcommand{\systemessansaccent}{Cordeuse}
\newcommand{\ilot}{2}
\newcommand{\ilotstr}{02}
\newcommand{\dossierilot}{\detokenize{Ilot_02 Cordeuse}}
\newcommand{\imageun}{Cordeuse}

\newcommand{\urlsysteme}{\href{https://www.costadoat.fr/systeme/48}{Ressources système}}
\newcommand{\videoavi}{\href{https://github.com/Costadoat/Sciences-Ingenieur/raw/master/Systemes/Cordeuse/Corder_raquette_de_tennis.avi}{Comment corder une raquette de tennis}}
\newcommand{\videoavii}{\href{https://github.com/Costadoat/Sciences-Ingenieur/raw/master/Systemes/Cordeuse/Demonstration_cordeuse.avi}{Démonstration de l'utilisation d'une cordeuse}}
\newcommand{\schemabloc}{Cordeuse_schema_bloc}
