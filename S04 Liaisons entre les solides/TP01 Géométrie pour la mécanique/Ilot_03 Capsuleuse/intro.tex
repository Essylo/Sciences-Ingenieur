\newcommand{\id}{150}
\newcommand{\nom}{Géométrie pour la mécanique}
\newcommand{\sequence}{04}
\newcommand{\num}{01}
\newcommand{\type}{TP}
\newcommand{\descrip}{Déterminer une fermeture géométrique et vérifier expérimentalement.}
\newcommand{\competences}{Mod2-C11: Modélisation géométrique et cinématique des mouvements entre solides indéformables}
\newcommand{\nbcomp}{1}
\newcommand{\systemes}{Capsuleuse}
\newcommand{\ilot}{3}
\newcommand{\imageun}{Capsuleuse}

\newcommand{\urlsysteme}{\href{https://www.costadoat.fr/systeme/50}{Ressources système}}
\newcommand{\matlabsimscape}{\href{https://github.com/Costadoat/Sciences-Ingenieur/raw/master/Systemes/Capsuleuse/Capsuleuse_Simscape.zip}{Modèle Simscape}}
\newcommand{\solidworks}{\href{https://github.com/Costadoat/Sciences-Ingenieur/raw/master/Systemes/Capsuleuse/Capsuleuse_Solidworks.zip}{Modèles Solidworks}}
\newcommand{\miseenoeuvre}{\href{https://github.com/Costadoat/Sciences-Ingenieur/raw/master/Systemes/Capsuleuse/Capsuleuse_MO/Capsuleuse_MO.pdf}{Mise en oeuvre de la capsuleuse}}
\newcommand{\edrawings}{\href{https://github.com/Costadoat/Sciences-Ingenieur/raw/master/Systemes/Capsuleuse/Capsuleuse.EASM}{Modèle eDrawings}}
