\newcommand{\id}{49}
\newcommand{\nom}{Analyse fonctionnelle}
\newcommand{\sequence}{01}
\newcommand{\num}{02}
\newcommand{\type}{TP}
\newcommand{\descrip}{Analyse du contexte de l'ingénierie. Mise en place d'une structure d'étude. Découverte et mise en œuvre des systèmes}
\newcommand{\competences}{A1-C1: Définitions normalisées \\ &  A2-C2: Description générale du système \\ &  A3-C3: Architecture générale d'un produit \\ &  A3-C4: Analyse d'architecture et de comportement \\ &  Com2-C4: Outils de communication}
\newcommand{\nbcomp}{5}
\newcommand{\systemes}{Barrière Sympact}
\newcommand{\systemessansaccent}{Barriere Sympact}
\newcommand{\ilot}{5}
\newcommand{\ilotstr}{05}
\newcommand{\dossierilot}{\detokenize{Ilot_05 Barrière Sympact}}
\newcommand{\imageun}{Barriere}

\newcommand{\urlsysteme}{\href{https://www.costadoat.fr/systeme/49}{Ressources système}}
\newcommand{\solidworks}{\href{https://github.com/Costadoat/Sciences-Ingenieur/raw/master/Systemes/Barriere Sympact/Barriere_Solidworks.zip}{Modèle Solidworks}}
\newcommand{\matlabsimscape}{\href{https://github.com/Costadoat/Sciences-Ingenieur/raw/master/Systemes/Barriere Sympact/Barriere_Simscape.zip}{Modèle Simscape}}
\newcommand{\edrawings}{\href{https://github.com/Costadoat/Sciences-Ingenieur/raw/master/Systemes/Barriere Sympact/Barriere.EASM}{Modèle eDrawings}}
\newcommand{\videoavi}{\href{https://github.com/Costadoat/Sciences-Ingenieur/raw/master/Systemes/Barriere Sympact/Utilisation_barriere_1.avi}{Utilisation de la barrière}}
\newcommand{\videoavii}{\href{https://github.com/Costadoat/Sciences-Ingenieur/raw/master/Systemes/Barriere Sympact/Utilisation_barriere_2.avi}{Utilisation de la barrière avec télécommande}}
\newcommand{\experimental}{\href{https://github.com/Costadoat/Sciences-Ingenieur/raw/master/Systemes/Barriere Sympact/Barriere_experimental.zip}{Analyse de résultats expérimentaux}}
\newcommand{\miseenoeuvre}{\href{https://github.com/Costadoat/Sciences-Ingenieur/raw/master/Systemes/Barriere Sympact/Barriere_MO/Barriere_MO.pdf}{Mise en oeuvre}}
\newcommand{\schemacinematique}{Barriere_cinematique}
