\newcommand{\id}{49}
\newcommand{\nom}{Analyse fonctionnelle}
\newcommand{\sequence}{01}
\newcommand{\num}{02}
\newcommand{\type}{TP}
\newcommand{\descrip}{Analyse du contexte de l'ingénierie. Mise en place d'une structure d'étude. Découverte et mise en œuvre des systèmes}
\newcommand{\competences}{A1-C1: Définitions normalisées \\ &  A2-C2: Description générale du système \\ &  A3-C3: Architecture générale d'un produit \\ &  A3-C4: Analyse d'architecture et de comportement \\ &  Com2-C4: Outils de communication}
\newcommand{\nbcomp}{5}
\newcommand{\systemes}{Moby Crea}
\newcommand{\ilot}{3}
\newcommand{\ilotstr}{03}
\newcommand{\dossierilot}{Ilot_03 Moby Crea}
\newcommand{\imageun}{Moby_Crea}

\newcommand{\urlsysteme}{\href{https://www.costadoat.fr/systeme/55}{Ressources système}}
\newcommand{\matlabsimscape}{\href{https://github.com/Costadoat/Sciences-Ingenieur/raw/master/Systemes/Moby Crea/mobycrea_complet.zip}{Modèle Simulink complet}}
\newcommand{\matlabsimscapei}{\href{https://github.com/Costadoat/Sciences-Ingenieur/raw/master/Systemes/Moby Crea/Mobycrea_Simscape.zip}{Modèle Simscape}}
\newcommand{\experimental}{\href{https://github.com/Costadoat/Sciences-Ingenieur/raw/master/Systemes/Moby Crea/MobyCrea_experimental.zip}{Analyse de résultats expérimentaux}}
\newcommand{\schemacinematique}{MobyCrea_cinematique}
