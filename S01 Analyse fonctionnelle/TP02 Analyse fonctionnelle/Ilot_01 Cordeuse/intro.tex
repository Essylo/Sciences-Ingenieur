\newcommand{\id}{49}
\newcommand{\nom}{Analyse fonctionnelle}
\newcommand{\sequence}{01}
\newcommand{\num}{02}
\newcommand{\type}{TP}
\newcommand{\descrip}{Analyse du contexte de l'ingénierie. Mise en place d'une structure d'étude. Découverte et mise en œuvre des systèmes}
\newcommand{\competences}{A1-C1: Définitions normalisées \\ &  A2-C2: Description générale du système \\ &  A3-C3: Architecture générale d'un produit \\ &  A3-C4: Analyse d'architecture et de comportement \\ &  Com2-C4: Outils de communication}
\newcommand{\nbcomp}{5}
\newcommand{\systemes}{Cordeuse}
\newcommand{\ilot}{1}
\newcommand{\ilotstr}{01}
\newcommand{\dossierilot}{\detokenize{Ilot_01 Cordeuse}}
\newcommand{\imageun}{Cordeuse}

\newcommand{\urlsysteme}{\href{https://www.costadoat.fr/systeme/48}{Ressources système}}
\newcommand{\videoavi}{\href{https://github.com/Costadoat/Sciences-Ingenieur/raw/master/Systemes/Cordeuse/Corder_raquette_de_tennis.avi}{Comment corder une raquette de tennis}}
\newcommand{\videoavii}{\href{https://github.com/Costadoat/Sciences-Ingenieur/raw/master/Systemes/Cordeuse/Demonstration_cordeuse.avi}{Démonstration de l'utilisation d'une cordeuse}}
\newcommand{\schemabloc}{Cordeuse_schema_bloc}
